The NetCDF (NCF) configuration utility can be activated by specifying the NCF6 option in a DIS or DISV input file.

The NCF configuration utility applies to NetCDF files created by \mf using model name file NETCDF\_MESH2D or NETCDF\_STRUCTURED keywords.  Although this utility is not required to create a NetCDF export, when it is configured it provides options related to data variable chunking, compression and grid mappings (projections).

\subsubsection{Structure of Blocks}
\lstinputlisting[style=blockdefinition]{./mf6ivar/tex/utl-ncf-options.dat}
\lstinputlisting[style=blockdefinition]{./mf6ivar/tex/utl-ncf-dimensions.dat}
\lstinputlisting[style=blockdefinition]{./mf6ivar/tex/utl-ncf-griddata.dat}
\vspace{5mm}

\subsubsection{Explanation of Variables}
\begin{description}
\input{./mf6ivar/tex/utl-ncf-desc.tex}
\end{description}
\vspace{5mm}

\subsubsection{Example Input File}
\lstinputlisting[style=inputfile]{./mf6ivar/examples/utl-ncf-example.dat}
