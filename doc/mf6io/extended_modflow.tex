Next to the standard \mf executable, a second, extended program executable is made available. This extended executable comes with additional functionality for which it partially relies on third-party libraries. Extended functionality includes the parallel computing capability and the use of NetCDF4 for input and output data. Because the external dependencies increase the complexity of the installation procedure, \mf will remain available in its standard set of functionality.

Extended \mf contains all features available in the standard executable, runs on the same input configuration, and produces the same results. Conversely, when running with the standard executable, some features described in this document (HPC Utility and NetCDF4 input and output, for example) will not be available and their configuration will be ignored or the program will terminate with an error. These features will be labeled accordingly below.

This sections describes input files that are only available with Extended \mf.  For more information on using Extended \mf, refer to the \href{https://github.com/MODFLOW-ORG/modflow6/wiki}{MODFLOW 6 Wiki} on the version-controlled \href{https://github.com/MODFLOW-ORG/modflow6}{MODFLOW 6 repository}.

\newpage
\subsection{NetCDF in Extended \mf}
Extended \mf can read and write version 4 NetCDF (Network Common Data Form) files. Two different formats are supported for each (input and output) file type: a structured format and a UGRID (Unstructured Grid) based Layered Mesh format. Each format aims to be cf-conventions compliant.

All \mf NetCDF files correspond to a model. Output files contain model dependent variable array(s). Input files can contain package gridded arrays that are supported as inputs from NetCDF files. These variables are indicated with red highlighting in package description chapters of this guide. Refer to the \mf Wiki (linked above) for additional documentation and details related to using NetCDF files as \mf inputs.

\mf structured NetCDF files support GWF, GWT and GWE models based on DIS package grids. To configure a structured NetCDF output file that includes the model dependent timeseries variable, add the NETCDF\_STRUCTURED option to the model name file:
\lstinputlisting[style=inputfile]{./mf6ivar/examples/ext-netcdf-gwf-structured.dat}

\mf UGRID Layered Mesh NetCDF files support GWF, GWT and GWE models based on DIS and DISV package grids. Each grid based array variable in these file types is split into a set of layer variables, including model dependent variable output. To configure a layered mesh NetCDF output file that includes model dependent timeseries variables, add the NETCDF\_MESH2D output file option to the model name file:
\lstinputlisting[style=inputfile]{./mf6ivar/examples/ext-netcdf-gwf-mesh.dat}

\newpage
\subsection{NetCDF (NCF) Configuration Utility}
The NetCDF (NCF) configuration utility can be activated by specifying the NCF6 option in a DIS or DISV input file.

The NCF configuration utility applies to NetCDF files created by \mf using model name file NETCDF\_MESH2D or NETCDF\_STRUCTURED keywords.  Although this utility is not required to create a NetCDF export, when it is configured it provides options related to data variable chunking, compression and grid mappings (projections).

\subsubsection{Structure of Blocks}
\lstinputlisting[style=blockdefinition]{./mf6ivar/tex/utl-ncf-options.dat}
\lstinputlisting[style=blockdefinition]{./mf6ivar/tex/utl-ncf-dimensions.dat}
\lstinputlisting[style=blockdefinition]{./mf6ivar/tex/utl-ncf-griddata.dat}
\vspace{5mm}

\subsubsection{Explanation of Variables}
\begin{description}
\input{./mf6ivar/tex/utl-ncf-desc.tex}
\end{description}
\vspace{5mm}

\subsubsection{Example Input File}
\lstinputlisting[style=inputfile]{./mf6ivar/examples/utl-ncf-example.dat}


\newpage
\subsection{High Performance Computing (HPC) Utility}
\input{utl_hpc.tex}

